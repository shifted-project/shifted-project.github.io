\section{Huber rings}

\begin{dfn}
	Let $A$ be a topological ring.
	\begin{enumerate}
		\item The ring $A$ is said to be \emph{nonarchimedean} if there is a neighbourhood basis of $0$ consisting of additive subgroups of $A$.
		\item The ring $A$ is said to be \emph{adic} if the subspaces $\{I^n : n \in \NN\}$ form a neighbourhood basis of $0$.
		In this case $I$ is said to be an \emph{ideal of definition} of $A$.
		\item The ring $A$ is said to be \emph{Huber} if it contains an open subring $B \subseteq A$ such that $B$ is adic for some finitely generated ideal of definition.
	\end{enumerate}
\end{dfn}

\begin{exm}
	Any ring equipped with the discrete topology is Huber.
	Any subring is a ring of definition, as it is $(0)$-adic.
\end{exm}

\begin{exm}
	Any adic ring that contains a finitely generated ideal of definition is Huber.
\end{exm}

\begin{exm}
	Let $B$ be a ring and $f \in B$ a nonzerodivisor.
	Topologise $A = B[f^{-1}]$ so that $\{f^nB : n\in \NN\}$ is a basis of open neighbourhoods of $0$.
	Now $A$ is Huber, $B$ is a ring of definition of $A$, and $(f)$ is an ideal of definition of $B$.
\end{exm}

\begin{dfn}
	Let $A$ be a topological ring.
	A subset $S \subseteq A$ is said to be \emph{bounded} if and only if, for any open neighbourhood $U$ of $0$, there exists an open neighbourhood $V$ of $0$ such that $VS \subseteq U$.
\end{dfn}

\begin{lem}
	If $A$ is a Huber ring, then a subring $B \subseteq A$ is a ring of definition if and only if it is open and bounded.
\end{lem}

\begin{dfn}
	Let $A$ be a topological ring.
	An element $f \in A$ is said to be \emph{topologically nilpotent} if $\lim_{n \to \infty} f^n = 0$.
	We write $A^{00} \subseteq A$ for the topologically nilpotent elements.

	A Huber ring $A$ is \emph{analytic} if the ideal generated by the topologically nilpotent elements contains $1$.
	It is \emph{Tate} if there exists a topologically nilpotent unit.
\end{dfn}

\begin{dfn}
	An element $f\in A$ is said to be \emph{power-bounded} if and only if $f^{\NN} \coloneq \{f^n : n \in \NN\}$ is bounded.
	We write $A^0 \subseteq A$ for the set of power-bounded elements.
\end{dfn}
